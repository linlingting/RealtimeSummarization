
\documentclass{article}
\usepackage{amsmath}
\usepackage{amsfonts}
\usepackage[linesnumbered,boxed]{algorithm2e}

\author{Lingting Lin, Yunjie Wang, Chen Lin}
\title{Realtime Event Summarization from Tweets with Replacements}
\begin{document}
\maketitle
\section{Introduction}
%Motivation: why do we need event summarization from tweets
Accidents, disasters, political rallies...we are eager to gather information about different kinds of live events that happen around us. In the past, we rely on experienced journalists to cover the stories. At now, thanks to the emergence of micro-blogging platforms, we are provided with more instant reports from various information sources. Study shows that the majority (over $85\%$) of trending topics in microblog sphere are headline news and real-life events~\cite{kwak2010twitter}. The massive amount of event related tweets highlights the importance of an event summarization system. 

%requirement: realtime
To facilitates knowledge management and improves user experiences, the event summary needs to be realtime. On one hand, the response must be realtime, which requires an efficient algorithm to analyze the tweets as they arrive. On the other hand, the summary must reflect the current status of the event, which indicates the summary should be updated if necessary. 


%Related Work: focus on extractive methods.
We have witnessed rapidly increasing popularity of research efforts in event summarization from tweets~\cite{}.  Most previous research works produce the summary in an extractive manner, i.e. they extract most representative tweets and blend them into a summary. A few researchers also mentioned the realtime factor in event summarizations~\cite{}. However, their summaries are incremental, i.e. recent summaries are addeded to cover new information, former summaries can not be deleted or replaced. The weakness is that the resulting summary can suffer from \textbf{poor brevity, inconsistency, and inadequacy to keep the readers up-to-date}.


%explain poor brevity, inconsistency, and inadequacy to deliver up-to-date information
There are three scenarios when a former summary needs to be deleted or replaced. (1) Better candidate. Due to the noisy nature of social sphere, it is well regarded that the selection of representative tweets is based on multiple criteria, i.e. the summary must achieve both good coverage and high quality. When the system detects a better tweet which covers the same topic, the former summary should be replaced to produce a brief summary. (2) Conflicting information. For many events users are interested on information such as number of injuries in a disaster, suspects in a crime and so on. The number of injuries could be changing as the event is evolving. The police might have suspected the wrong person. If the former summary is kept, the sumamry will be inconsistent. (3) Lost of public attention to a historical stage. When the event is moving into a new epoch, there is no need to maintain outdated summaries for the historical stage which we are over.

%Re-computation is infeasible (1) streaming text (2) realtime response
In this paper we study the problem of realtime event summarization. We focus on designing algorithms that can efficiently replace former summaries with approapriate substitues. To the best of our knowledge, this has not been explicitly studied before. There are two challenges arised in doing so.

Firstly, at the macro-level, we are handling social text streams. Generating summaries involves of peer-to-peer similarity computation. However, as the amount of available tweets constantly increases, recomputation based on a full similarity matrix is infeasible. Our goal is to incorporate new tweets as it becomes available, and to limit the storage and processing of old tweets. 

%Another challenge: similarity content + structure
Secondly, at the micro-level, determining whether a tweet needs to be replaced is computationally expensive. For example, to decide whether two tweets are conflicting, we have to extract the grammar structure of each tweet and match the corresponding compartments. To speed up the summarization, we have to reduce the volumn of tweets that need to be deep analyzed.   

%present work
To address the above two challenges, we embed the micro-level deep analysis into a macro-level streaming algorithm. We first consider the case a static summary is generated from a set of tweets. We model it as an integer programing problem and solve it by the simplex method on a relaxed linear form. We then modify the simplex method to deal with streaming tweets. We detect old summaries that are most likely to be conflicting with new summaries in the simplex iterations to reduce computational cost of micro-level analysis. 


%paper structure

\section{Related Work}
%Event summarization

\section{Static Summarization}
\subsection{Problem Definition}
Suppose that we have $N$ tweets to be summarized, within which $M$ tweets are credible and relevant. We are going to select a few representative posts from the tweet universe to form the summary. To model this problem, we use a vector $\tilde{\mathbf{x}}\in R^N$ , where each element $\tilde{\mathbf{x}}_j\in \{0,1\}$ is a binary variable. If a tweet $i$ is chosen as the abstractive sentence, the corresponding $\tilde{\mathbf{x}}_i=1$ . Otherwise, we set  $\tilde{\mathbf{x}}_i=0$. We use another $N$-dim vector $\tilde{\mathbf{c}}\in R^M$ to describe the loss of choosing each tweet as a candidate. $\tilde{\mathbf{A}}\in R^{M\times N}$  is a similarity matrix, where $\tilde{a}_{i,j}$  is the similarity between a credible and relevant tweet $i$ and a candidate tweet $j$ in the tweet universe. $\mathbf{b}\in R^{M}$ is a weight vector, where $b_{i}$ indicates the importance of $i$ being covered in the summary. Our objective is to

\begin{equation*}
\min \tilde{\mathbf{c}}^T\tilde{\mathbf{x}}\textrm{ subject to } \tilde{\mathbf{A}}\tilde{\mathbf{x}}\geq \mathbf{b}, \tilde{\mathbf{x}}\in \{0,1\}
\end{equation*}

\subsection{Methodology Overview}
We first transform it to a standard form of bounded linear programming problem by making the following adjustments: $\mathbf{c}=[\tilde{\mathbf{c}},\mathbf{0}],\mathbf{x}=[\tilde{\mathbf{x}},\mathbf{z}]^T,\mathbf{A}=[\tilde{\mathbf{A}},-\mathbf{I}]$, where $\mathbf{I}$ is the $M\times M$ identity matrix. Therefore we have

\begin{equation}
\min \mathbf{cx} \textrm{ subject to } \mathbf{Ax} = \mathbf{b}, \mathbf{x}\geq \mathbf{0},\tilde{\mathbf{x}} \leq \mathbf{1}
\end{equation}

We modify the LP so that there is an easy choice of basic solution. We start by solving

\begin{equation}
\min \mathbf{e^{T}s} \textrm{ subject to } \mathbf{Ax} + \mathbf{Is} = \mathbf{b}, \mathbf{x}\geq \mathbf{0},\tilde{\mathbf{x}} \leq \mathbf{1},\mathbf{s}\geq \mathbf{0}
\end{equation}

where $\mathbf{e}$ is the vector of all ones, $\mathbf{s}$ are called artificial variables,and $b_{i}\geq \mathbf{0}$.
To solve the above LP relaxation, we adopt the bounded simplex method. In each iterate, only $M$ variables are selected as the basic points. Suppose $\mathbf{A}_{,j}$ is the $j$th column, and the corresponding columns of basic variables in $\mathbf{A}$ are denoted as $\mathbf{A_B}$,others are denoted as $\mathbf{A_N}$. Non-basic variables are either at upper bound ($j \in U$) or at lower bound ($j\in L$).

Define $\mathbf{c^{'}}=[\mathbf{e},\mathbf{0}], \mathbf{x}^{'} = [\mathbf{x},\mathbf{s}]^T$ and $\mathbf{A}^{'} = [\mathbf{A},\mathbf{I}]$ so that the constraints of the modified LP can be written as $\mathbf{A}^{'}\mathbf{x}^{'} = \mathbf{b}, \mathbf{x}^{'} \geq \mathbf{0},\tilde{\mathbf{x}} \leq \mathbf{1}$.

Step 1: Let $\mathbf{B}$ be the indices of the artificial variables and set the all non-basic variables to be $x_j=0$. Then $\mathbf{B}$ is a basis,since the corresponding columns of $\mathbf{A_B^{'}}$ are $\mathbf{I}$, the identity,the corresponding basic feasible solution is $\mathbf{x} = \mathbf{0}, \mathbf{z} = \mathbf{b}$.

Step 2: Pricing. $\mathbf{y} = {\mathbf{B}^T}^{-1}\mathbf{e_B}$

Step 3: Compute $\bar{c_j} = c_j - y^T\mathbf{A}_{,j}^{'}$. If $x_j = 0, \bar{c_j} > 0$ and $x_j = 1,\bar{c_j} < 0$, then the problem is solved. Else pick $q$ the most contradictive variable, i.e. the most negative $\bar{c_q}$ for $x_q = 0$ as the entering index(step 4 in Algorithm 1).

Step 4: Choose the outing index(step 5 to step 27 in Algorithm 1) and update the basis and the value of variables.

Step 5: If all artificial variables are non-basic or some artificial variables are in the basis but all $x_{z_i} = 0$ then go to Step 6,otherwise return to Step 2.

Step 6: If some artificial variables are in the basis,remove them from basis. Change $\mathbf{c^{'}}=[\tilde{\mathbf{c}},\mathbf{0}]$ , repeat Step 2 to Step 4.

\begin{algorithm}
\caption{the bounded simplex method}
\KwIn{$\mathbf{c}=[\mathbf{e},\mathbf{0}], \mathbf{x}^{'} = [\mathbf{x},\mathbf{s}]^T$ and $\mathbf{A}^{'} = [\mathbf{A},\mathbf{I}]$}
\KwOut{$\mathbf{x_B^{'}} = \mathbf{A_B^{'}}^{-1}(\mathbf{b} - \mathbf{u})$, where $\mathbf{u} = \Sigma_{j\in U} \mathbf{A}_{,j}^{'}$}

%$\mathbf{B} \leftarrow$ \{the indices of the artificial variables\}\;
%$\mathbf{x_N}$ = 0\;
Pricing $\mathbf{y} = {\mathbf{B}^T}^{-1}\mathbf{e_B}$\;
Compute $\bar{c_j} = c_j - y^T\mathbf{A}_{,j}^{'}$\;
\While{$\exists x_j = 0, \bar{c_j} < 0$ or $x_j = 1,\bar{c_j} < 0$}
{
    $q = \arg\max_{j} \left\{\bar{c_j} \forall j\in\{\mathbf{z_N}, \mathbf{s_N}, \mathbf{x_L}  \},-\bar{c_j} \forall j\in\mathbf{x_U}  \right\}$\;
    $d = \mathbf{B}^{-1}\mathbf{A}_{,q}^{'}$\;
    \If{$q \in \{ \mathbf{z_N}, \mathbf{s_N}\}$}
    {
        $x_q^{new} = \min_i \left\{\{\frac{x_{\mathbf{B}_{i}}}{d_i}, \frac{s_{\mathbf{B}_{i}}}{d_i}, \frac{z_{\mathbf{B}_{i}}}{d_i}\} \forall d_i>0,\{\frac{x_{\mathbf{B}_{i}}-1}{d_i}\} \forall d_i<0\right\}$\;
        $\mathbf{x}_{\mathbf{B}^{old}}^{'new} = \mathbf{x}_{\mathbf{B}^{old}}^{'old} - dx_q^{new}$\;
        $p=\arg\min_{i}$\;
        $\mathbf{B}^{new} \leftarrow \mathbf{B}^{old} - \{p\} \bigcup \{q\}$\;
    }
    \If{$q \in L$}
    {
        $x_q^{new} = \min_i \left\{\{\frac{x_{\mathbf{B}_{i}}}{d_i}, \frac{s_{\mathbf{B}_{i}}}{d_i}, \frac{z_{\mathbf{B}_{i}}}{d_i}\} \forall d_i>0, \{\frac{x_{\mathbf{B}_{i}}-1}{d_i}\} \forall d_i<0, 1 \right\}$\;
        $\mathbf{x}_{\mathbf{B}^{old}}^{'new} = \mathbf{x}_{\mathbf{B}^{old}}^{'old} - dx_q^{new}$\; \If{$x_q^{new} \neq 1$}
        {
            $p=\arg\min_{i}$\;
            $\mathbf{B}^{new} \leftarrow \mathbf{B}^{old} - \{p\} \bigcup \{q\}$\;
        }
    }
    \If{$q \in U$}
    {
        $x_q^{new} = 1 - \min_i \left\{\{\frac{x_{\mathbf{B}_{i}}}{-d_i}, \frac{s_{\mathbf{B}_{i}}}{-d_i}, \frac{z_{\mathbf{B}_{i}}}{-d_i}\} \forall d_i<0, \{\frac{1-x_{\mathbf{B}_{i}}}{d_i}\} \forall d_i>0, 1 \right\}$\;
        $\mathbf{x}_{\mathbf{B}^{old}}^{'new} = \mathbf{x}_{\mathbf{B}^{old}}^{'old} + d(1-x_q^{new})$\;
        \If{$x_q^{new} \neq 0$}
        {
            $p=\arg\min_{i}$\;
            $\mathbf{B}^{new} \leftarrow \mathbf{B}^{old} - \{p\} \bigcup \{q\}$\;
        }
    }
}

\end{algorithm}


\section{Update Summarization}

%first show that we don't need A by proving the constraint 1 hold by itself for all scenarios (conflicting, outdate, better converage)
\begin{equation}
\min c_0x_0+c_1x_1\\
s.t. \begin{bmatrix}
A & D \\
D^T & B
\end{bmatrix}\begin{bmatrix}
 x_0\\
x_1
\end{bmatrix}\geq \begin{bmatrix}
b_0\\
b_1
\end{bmatrix}
\end{equation}

%check whether constraint 2 is satisfied, note that D is small due to the sparseness of x_0
$D^Tx_0\geq b_1$

%if b_1 largely not covered, outdated, change constraint


%else implement simplex

%if an index gets out, check if its conflicting by measuring structural similarity


\end{document}
